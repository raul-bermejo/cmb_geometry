The Cosmic Microwave Background (CMB) is the first electromagnetic signal we can study after the Big Bang. It thus encodes a wealth of cosmological information from the epoch of recombination (and reionization). In this study, we use data from the Planck satellite mission to measure the anisotropy of the CMB. Studying the statistical fluctuations in this primordial signal allows us to infer the geometry of the Universe.

%\vspace{0.4em}
%\begin{figure}
%\begin{minipage}{0.43\textwidth}
%	\centering\includegraphics[width=0.85\textwidth]{poster_labyCMB/img/PlanckCMB_commander.png}
%	\caption{CMB as measured by the Planck satellite \cite{planck:2018_iv}.}
%	\label{fig:esa_cmb}
%\end{minipage}
%\hspace{1em} 
%\end{figure}

\vspace{0.5em}
\begin{figure}
	\begin{minipage}{0.3\textwidth}
		\centering\includegraphics[width=1.1\textwidth]{poster_labyCMB/img/sphere.png}
	\end{minipage}
	\hspace{1em}
	\begin{minipage}{0.3\textwidth}
	    \captionsetup{width=2\textwidth}
		\centering\includegraphics[width=1\textwidth]{poster_labyCMB/img/flat.png}
		\hspace{14em}\caption{Different geometries that the universe can take. From left to right: positively curved (closed $k=1$), no curvature (flat $k=0$) and negatively curved (open $k=-1$).}
		\label{fig:geometries}
	\end{minipage}
	\hspace{1em}
	\begin{minipage}{0.3\textwidth} 
		\centering\includegraphics[width=1\textwidth]{poster_labyCMB/img/saddle.png}
	\end{minipage}
\end{figure}

In the standard $\Lambda$CDM model of cosmology [add reference?], the geometry of the Universe is encoded in the \textit{curvature density parameter}, $\Omega_k$:

\begin{equation}
    \Omega_{k}=-\frac{k c^{2}}{a^{2} H^{2}},
\end{equation}
\hspace{1em} 

where $k$ corresponds to the curvature of the Universe, $c$ is the speed of light, $a$ is the expansion scale factor, and $H = \frac{\dot{a}}{a}$ is the Hubble-Lemaître parameter. $\Omega_k$ indicates the energy contribution to the Universe due to its curvature, and $k = −1, +1, 0$ corresponds to an open, closed or flat Universe respectively, as shown in figure ~\ref{fig:geometries}.

By comparing the CMB signal from that of simulated Universes with different values of $\Omega_k$ (also known as Boltzmann codes, see \cite{lesgourgues:2011}), we can infer the value of $\Omega_k$ corresponding to our Universe. 