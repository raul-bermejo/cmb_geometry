The Cosmic Microwave Background (CMB) is the first electromagnetic signal we can study after the Big Bang. It encodes a wealth of cosmological information from the epoch of recombination, reionization, and cosmological parameters are encoded in it. In this study, we only focus on studying the temperature fluctuations of this primordial signal to measure the Universe's geometry. More specifically, we study the CMB anisotropy using data from the Planck satellite mission to infer the curvature density parameter.

\vspace{1.5cm}
\begin{figure}
\subfloat $k\ =\ +1$ (closed) {\includegraphics[width =.2\textwidth]{poster_labyCMB/img/sphere.png}} \hspace{1.5cm} 
\subfloat {\includegraphics[width =.25\textwidth]{poster_labyCMB/img/saddle.png}} $k\ =\ -1$ (open)\\ \vspace{1.5cm} \hspace{5cm}
\subfloat {\includegraphics[width =.3\textwidth]{poster_labyCMB/img/flat.png}} $k\ =\ 0$ (closed) \vspace{1cm}
\caption{Geometries the Universe can theoretically take based on $k=+1, \ -1, \ 0$ (positive curvature, negative curvature and no curvature respectively).}
\label{fig:geometries}
\end{figure}


In the standard $\Lambda$CDM model of cosmology, the geometry of the Universe is encoded in the \textit{curvature density parameter}, $\Omega_k$:

\begin{equation}
    \Omega_{k}=-\frac{k c^{2}}{a^{2} H^{2}},
\end{equation}
\vspace{1em} 

where $k$ corresponds to the curvature of the Universe, $c$ is the speed of light, $a$ is the expansion scale factor, and $H = \frac{\dot{a}}{a}$ is the Hubble-Lemaître parameter. $\Omega_k$ indicates the energy contribution to the Universe due to its curvature, and figure ~\ref{fig:geometries} depicts the different curvature $k = −1, +1, 0$ can take.
\vspace{1em} 

By comparing the CMB signal measured by the Planck satellite from that of simulated Universes with different values of $\Omega_k$ (also known as Boltzmann codes, see \cite{lesgourgues:2011}), we can infer the value of $\Omega_k$ corresponding to our Universe. 