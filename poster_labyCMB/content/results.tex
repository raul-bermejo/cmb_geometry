\begin{figure}
	\begin{minipage}{0.95\textwidth}
		\centering\includegraphics[width=0.85\textwidth]{poster_labyCMB/img/Planck_Cl_143x143__60perc.png}
		\caption{Computed cross-power spectrum from the Planck satellite.}
		\label{fig:planck_cls}
	\end{minipage}
\end{figure}

As we are only interested in the first peak of the cross power spectrum, we fitted the computed $C_{\ell}$ for the Planck data using the following model:

\begin{equation}
    C_{\ell} = a(\ell-\ell_{peak})^2 + b,
\end{equation}

where $a, \ell_{peak}$ and $b$ are model parameters. Figure \ref{fig:planck_cl} shows the fit for the model, with a multipole value of $\ell_{peak} = 211.14 \pm 1.10$.

Finally, using the curves in figure \ref{fig: boltz_Cls} and our estimated $\ell_{peak}$, we performed a one-dimensional fit to find an estimate of $\Omega_k$. Our estimate of the curvature density parameter as measured by the Planck satellite data corresponds to 

\begin{equation*}
    \Omega_k = -0.028 \pm 0.012.
\end{equation*}

This result is within a $2\sigma$ accordance with respect to literature values from the Planck 2018 release \cite{planck:2018_main}, thus suggesting a flat geometry for our Universe. 

	
