We infer the geometrical signal from the Planck data by studying the fluctuations in temperature $\Delta T$ in the sky maps. These fluctuations in the signal can be measured by the so-called \textit{cross power spectrum} $C_{\ell}$. An unbiased estimator for $C_{\ell}$ \cite{tristram:2005} is given by 

\begin{equation}
\widehat{C_{\ell}}=\sum_{m=-\ell}^{\ell} \frac{\left|a_{\ell m}\right|^{2}}{2 \ell+1},
\end{equation}
\vspace{1em}

where $\ell$ corresponds to the multipole (associated with an angular size) and $a_{\ell}$ corresponds to the Fourier coefficients of the spherical harmonic decomposition of $\Delta T$:
\begin{equation}
a_{\ell m}=\int \Delta T(\hat{n}) Y_{\ell m}(\hat{n}) d \Omega.
\end{equation}
\vspace{1em}

We compute $C_{\ell}$ using the open-source Python package \textsc{healpy} (\cite{healpy:2019}; \cite{healpix:2005}). Next, to reduce the correlations and errors in $C_{\ell}$ induced by the cut sky, we bin the cross power spectrum. Following \cite{hivon:2002}, we choose the 'flattened spectrum' as our binning such that $C_{\ell} \equiv \ell(\ell+1) C_{\ell} / 2 \pi$.


\vspace{0.2em}
\begin{figure}
	\begin{minipage}{.94\textwidth}
		\centering\includegraphics[width=.75\textwidth]{poster_labyCMB/img/BoltzCode_Cls.png}
		\caption{Cross power spectra for simulated Universes with different curvature density parameters $\Omega_k$.}
		\label{fig: boltz_Cls}
	\end{minipage}
\end{figure}
\vspace{0.4em}


The first peak of $C_l$ encodes the information about $\Omega_k$ [add reference!]. From figure \ref{fig: boltz_Cls}, one can see that different values of $\Omega_k$ shift the value of $\ell_{peak}$.

Thus to infer $\Omega_k$, we fit the value $\ell_{peak}$ from the Planck signal to that of the simulated Universes
we infer the computed $\C_{\ell}$ from the Planck sky map to that of a set of simulated Universes (computed using the Python package \textsc{CAMB} \cite{camb:2011}).
