The Planck satellite sky maps used for this study are publicly available at \href{http://pla.esac.esa.int/pla/#maps}{ESA's Planck mission website}. We used two half-mission single-frequency maps of 143 $GHz$.

\vspace{0.5em}
\begin{figure}
	\begin{minipage}{0.3\textwidth}
		\centering\includegraphics[width=1.1\textwidth]{poster_labyCMB/img/planck_sky0.png}
	\end{minipage}
	\hspace{1em}
	\begin{minipage}{0.3\textwidth}
	    \captionsetup{width=2\textwidth}
		\centering\includegraphics[width=1\textwidth]{poster_labyCMB/img/planck_sky1.png}
		\hspace{14em}\caption{Different geometries that the universe can take. From left to right: positively curved (closed $k=1$), no curvature (flat $k=0$) and negatively curved (open $k=-1$).}
		\label{fig:geometries}
	\end{minipage}
	\hspace{1em}
	\begin{minipage}{0.3\textwidth} 
		\centering\includegraphics[width=1\textwidth]{poster_labyCMB/img/planck_sky2.png}
	\end{minipage}
\end{figure}

To reduce the signal-to-noise ratio, we applied two masks to each of the sky maps:

\begin{enumerate}

	\item A mask with a galactic cover of $f_{sky} = 60 \%$ to reduce the noise due to HI galactic emission.
	
 	\item A point source mask to reduce the noise due to the satellite instrument.
 	
\end{enumerate}

Figure ~\ref{fig: cmb_maps} shows an example of the effect the masks have on one of the sky maps. The last map in ~\ref{fig: cmb_maps} shows a less populated signal than the original one. After applying both of these masks to each sky map, the datasets were ready for the statistical analysis.



